\chapter{Conclusion and Future work}
\section{Conclusion}
Encapsulating text requires identifying the topic of the document and optimizing the summary such that minimum words are used to represent maximum information. The opposite
relationship between length(number of words) and information retention makes this task difficult. Results achieved by incorporating simple techniques based on information
rich sentence location(position hypothesis), and frequency of text features(proper noun phrases), indicate that some text segments (sentences, and phrases) can represent the entire document. 
These two parameters (position hypothesis, and frequency of text features) frequently conclude by selecting two or more sentences. Method adopted to generate 
encapsulation relies on tracing, and extracting a specific pattern (chunks corresponding to noun, verb, noun phrases) within sentence, therefore the length of encapsulation
grows with the number of sentences extracted. On the other hand, utilizing only features and constructing encapsulation(a three words sentence - subject, verb, and object) 
requires a sophisticated algorithm that could recognize and relate text segments. \\
Some key observations on evaluation and results include the following:-
\begin{enumerate}
 \item Few sentences completely represent the information contained within the document, extracting the noun, verb, noun pattern results in a concise and readable 
phrase but ignoring other parts of the sentence causes encapsulation to be incomplete.
 \item Sentences of the corpus span multiple clauses, containing subject, verb and object at unusual positions. Encapsulation method would not come up with a
readable phrase for such sentences.
\end{enumerate}

\section{Future work}
This results obtained through experiments have showed that frequent text features can assist in identifying important information for generating concise text 
representation, however more corpus specific features (e.g. text in quotation marks, facts presented as numeric values) and cue phrases can be used in addition to
improve topic identification. \\
In order to improve encapsulations two possibilities can be explored:-
\begin{enumerate}
 \item classification of sentences in different categories so that each type is dynamically encapsulated using appropriate method, rather than statically using
 noun-verb-noun patterns
 \item identifying text segments based on new features as mentioned above that must be retained to provide more complete and readable encapsulations. 
 \item identifying a reliable data set for evaluating encapsulation.
\end{enumerate}
Incorporating methods that can discover semantics in text, and smart sentence compression algorithm can further improve the system.
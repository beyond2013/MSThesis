\chapter{Literature Review}
Literature on generating and evaluating summaries (using extraction)  includes a variety of techniques and algorithms. Although the approaches to summarize 
text utilize different data structures and corpus, most of them try to identify key features present in text. Generally, all systems proposed for extractive summarization
encompass the process of selecting key terms, assigning weights to term, aggregating term weights to obtain sentence weight, and rearranging the sentences 
to produce summary.

\citeA{kupiec-train} describe a method that is able to learn from data. The classification function categorizes each sentence as worthy of extraction 
or not, using a naive-Bayes classifier. Let s be a particular sentence, S the set of sentences that make up the summary, and $F_{1} , . . . , F_{k}$ the features.
Assuming independence of the features:
 \begin{displaymath}
  P(s \in S \mid F_{1},F_{2} \ldots F_{k}) = \frac{\Pi_{i=1}^k P(F_{i} \mid s\in S ) . P(s\in S)}{\Pi_{i=1}^k P(F_{i})}
 \end{displaymath}
Each sentence was scored according to the above equation, and top n sentences were extracted. A corpus of technical documents was used to evaluate the system.
The authors manually analyzed the match between the manual abstracts and the actual document and created a mapping. This mapping was used to evaluate auto-extracts.\\

\citeA{Rama-Structure} 
emphasized the importance of document structure, and presented a method to create query-specific summaries by extracting associations between their
fragments. They represented a document by a weighted graph, nodes of the graph are text fragments(paragraphs), weights were assigned based on traditional Information
Retrieval(IR) weighting formulas. They suggested the summary of a document to be the minimum spanning tree on the corresponding document graph that contains all the 
keywords or equivalent based on thesaurus. For generating query specific summaries, each node of the precomputed document graph for a document was scored according
to IR ranking functions. The best summary was obtained by the top-ranked spanning tree that possessed all the keywords.
The ranking considered both the node and the edge weight.\\

\citeA{Nenkova-newswire}
 published an overview of the results achieved in different type of summarization tasks(single document summarization, multi document summarization,
summarization focused by question and headline generation). Results of his analysis of variance model revealed that most progress has been made in generic multi document
summarization and the most challenging task is to produce user focused summary in answer to a question or topic. He also discovered from the analysis that single document
summarization systems are further from human performance, than in the multi document summarization systems. He pointed out that only coverage and few other metrics have 
been used to evaluate summarization systems, with little attention paid on coherence and cohesion. He remarked that no systematic study had been done to identify types 
of input (whether easy, or hard); which could assists in development of more specialized summarizers.\\

\citeA{Svore-RankNet}
 proposed an algorithm based on Neural Nets and the use of third party datasets to deal with the problem of extractive summarization, outperforming 
the baseline with statistical significance. The authors used a dataset containing 1365 documents gathered from CNN.com,each consisting of the title, time stamp, three or 
four human generated story highlights and the article text. They considered the task of creating three machine highlights. The human generated highlights were not 
verbatim extractions from the article itself. The authors evaluated their system using two metrics: the first one concatenated the three highlights produced by the system,
concatenated the three human generated highlights, and compared these two blocks; the second metric considered the ordering and compared the sentences on an individual 
level.\\

\citeA{siddiqui-multilingual}
 described language independent, generic, extractive summarization algorithm for single document. For determining content rich sentences 
they constructed various(noun/proper noun, document feature, and enhanced theme feature) vectors. They suggested that noun and proper noun are content rich and easy to find. 
document feature vector was constructed using the threshold as 30\% of the highest frequency. To determine the central theme of the document the authors proposed the theme 
feature vector which contains the union of frequent terms from the title and frequent proper nouns. Sentence weight is calculated as a summation of document feature vector,
and theme feature vector. The sentence weight was computer by aggregating those sentence terms, which belong to the document feature vector and the theme feature vector. 
The risk of extracting incomplete sentence(sentence referring to previous sentence) was tackled by computing Sentence Reference Index (SRI). Authors assumed that
important information may exist anywhere in the document and divided the document into number of partitions depending upon the length of summary to be 
generated. From each partition highest weight sentences were selected. If the sentence made reference to previous sentence and the length of summary allowed
inclusion, previous sentence was included in the summary otherwise it was dropped. Some post processing was also applied on the summary extracted by adding or
dropping sentences in order to make it coherent. The evaluation of suggested algorithm produced satisfactory results equally for English documents as well as 
Hindi, Gujarati and Urdu. \\

\citeA{Ledeneval-FreqSequences}
 showed that words that are parts of bigrams that repeat more than once in text are good terms to describe the text's contents. 
They also emphasized that only the maximal frequent sequences bear important meaning and can express ideas both important and specific for the document. Authors adopted
a very straight forward and logical algorithm for their experiments. In first step key terms were selected to decide which features are to be used to describe the 
sentences. Next, term weights were calculated to decide how important each feature is. Sentence weight was obtained by combining the key term weights for the sentence. 
Sentence selection involved deciding which sentences should be selected as part of summary. The proposed method outperformed most state of the  art methods. 
For few experiments the results were even better than the baseline method(first line of the first paragraph). 

\clearpage 
